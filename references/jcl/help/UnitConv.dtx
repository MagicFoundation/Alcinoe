@@UnitConversions.AngleConversion
<GROUP UnitConversions>
<TITLE Angle Conversion>
<TOPICORDER 100>
--------------------------------------------------------------------------------
@@UnitConversions.CoordinateConversion
<GROUP UnitConversions>
<TITLE Coordinate Conversion>
<TOPICORDER 200>
--------------------------------------------------------------------------------
@@UnitConversions.LengthConversion
<GROUP UnitConversions>
<TITLE Length Conversion>
<TOPICORDER 300>
--------------------------------------------------------------------------------
@@UnitConversions.MassConversion
<GROUP UnitConversions>
<TITLE Mass Conversion>
<TOPICORDER 400>
--------------------------------------------------------------------------------
@@UnitConversions.Power
<GROUP UnitConversions>
<TITLE Power>
<TOPICORDER 500>
--------------------------------------------------------------------------------
@@UnitConversions.PressureConversion
<GROUP UnitConversions>
<TITLE Pressure Conversion>
<TOPICORDER 600>
--------------------------------------------------------------------------------
@@UnitConversions.TemperatureConversion
<GROUP UnitConversions>
<TITLE Temperature Conversion>
<TOPICORDER 700>
--------------------------------------------------------------------------------
@@UnitConversions.Velocity
<GROUP UnitConversions>
<TITLE Velocity>
<TOPICORDER 800>
--------------------------------------------------------------------------------
@@UnitConversions.VolumeConversion
<GROUP UnitConversions>
<TITLE Volume Conversion>
<TOPICORDER 900>
--------------------------------------------------------------------------------
@@UnitConversions.Miscellaneous
<GROUP UnitConversions>
<TITLE Volume Conversion>
<TOPICORDER 1000>
--------------------------------------------------------------------------------
@@MakePercentage
<GROUP UnitConversions.Miscellaneous>
Summary:
  Hehe, no way...
Description:
  I dare not describe this function's purpose. See the Notes section below. As a
  tribute to that wonderful article this function has been designated the alias
  HowAOneLinerCanBiteYou. You are encouraged to use that alias in your applications
  as opposed to the much more readable and self-explanatory name MakePercentage :-)
Parameters:
  Step - Sorry, no can do.
  Max - No idea, but I do know that passing 0 will trigger an assertion if you have assertions enabled or a runtime error (EDivByZero) if not.
Result:
  All I have to say about this is that it is very well possible that the result
  may be larger than 100 and that the sign of the arguments is preserved so it is
  possible that the result is negative.
Notes:
  For a very educational dissection of this function have a look at the article written by Robert Marquardt entitled " Lessons in good programming or how a one liner can bite you" available at the Voyager section on the JEDI website.
Donator:
  Marcel van Brakel
--------------------------------------------------------------------------------
@@HowAOneLinerCanBiteYou
<ALIAS MakePercentage>
--------------------------------------------------------------------------------
@@CelsiusToKelvin
<GROUP UnitConversions.TemperatureConversion>
Summary:
  Converts a temperature from celsius to kelvin.
Description:
  The CelsiusToKelvin routine converts the specified temperature from degrees Celsius to degrees Kelvin.
Parameters:
  T - The temperature, in Celsius, to convert to Kelvin.
Result:
  The temperature after conversion to degrees Kelvin.
Donator:
  Marcel van Brakel
--------------------------------------------------------------------------------
@@CelsiusToFahrenheit
<GROUP UnitConversions.TemperatureConversion>
Summary:
  Converts a temperature from Celsius to Fahrenheit.
Description:
  The CelsiusToFahrenheit routine converts the specified temperature from degrees Celsius to degrees Fahrenheit.
  Have a look at the source. It has been implemented to be an example of selfdocumenting code.
Parameters:
  T - The temperature, in Celsius, to convert to Fahrenheit.
Result:
  The temperature after conversion to degrees Fahrenheit.
Donator:
  Marcel van Brakel
--------------------------------------------------------------------------------
@@KelvinToCelsius
<GROUP UnitConversions.TemperatureConversion>
Summary:
  Converts a temperature from Kelvin to Celsius.
Description:
  The KelvinToCelsius routine converts the specified temperature from degrees Kelvin to degrees Celsius.
Parameters:
  T - The temperature, in Kelvin, to convert to Celsius.
Result:
  The temperature after conversion to degrees Celsius.
Donator:
  Marcel van Brakel
--------------------------------------------------------------------------------
@@KelvinToFahrenheit
<GROUP UnitConversions.TemperatureConversion>
Summary:
  Converts a temperature from Kelvin to Fahrenheit.
Description:
  The KelvinToFahrenheit routine converts the specified temperature from degrees Kelvin to degrees Fahrenheit.
Parameters:
  T - The temperature, in Kelvin, to convert to Fahrenheit.
Result:
  The temperature after conversion to degrees Fahrenheit.
Donator:
  Marcel van Brakel
--------------------------------------------------------------------------------
@@FahrenheitToCelsius
<GROUP UnitConversions.TemperatureConversion>
Summary:
  Converts a temperature from Fahrenheit to Celsius.
Description:
  The FahrenheitToCelsius routine converts the specified temperature from degrees Fahrenheit to degrees Celsius.
  Have a look at the source. It has been implemented to be an example of selfdocumenting code.
Parameters:
  T - The temperature, in Fahrenheit, to convert to Celsius.
Result:
  The temperature after conversion to degrees Celsius.
Donator:
  Marcel van Brakel
--------------------------------------------------------------------------------
@@FahrenheitToKelvin
<GROUP UnitConversions.TemperatureConversion>
Summary:
  Converts a temperature from Fahrenheit to Kelvin.
Description:
  The FahrenheitToKelvin routine converts the specified temperature from degrees Fahrenheit to degrees Kelvin.
Parameters:
  T - The temperature, in Fahrenheit, to convert to Kelvin.
Result:
  The temperature after conversion to degrees Kelvin.
Donator:
  Marcel van Brakel
--------------------------------------------------------------------------------
@@CycleToDeg
<GROUP UnitConversions.AngleConversion>
Summary:
  Converts an angle from cycles to degrees.
Description:
  The CycleToDeg routine converts the specified angle, in cycles, to the corresponding angle in degrees.
Parameters:
  Cycles - The angle, in cycles, to convert to degrees.
Result:
  The angle after conversion to degrees.
Donator:
  Marcel van Brakel
--------------------------------------------------------------------------------
@@CycleToGrad
<GROUP UnitConversions.AngleConversion>
Summary:
  Converts an angle from cycles to gradians.
Description:
  The CycleToGrad routine converts the specified angle, in cycles, to the corresponding angle in gradians.
Parameters:
  Cycles - The angle, in cycles, to convert to gradians.
Result:
  The angle after conversion to gradians.
Donator:
  Marcel van Brakel
--------------------------------------------------------------------------------
@@CycleToRad
<GROUP UnitConversions.AngleConversion>
Summary:
  Converts an angle from cycles to radians.
Description:
  The CycleToRad routine converts the specified angle, in cycles, to the corresponding angle in radians.
Parameters:
  Cycles - The angle, in cycles, to convert to radians.
Result:
  The angle after conversion to radians.
Donator:
  Marcel van Brakel
--------------------------------------------------------------------------------
@@DegToCycle
<GROUP UnitConversions.AngleConversion>
Summary:
  Converts an angle from degrees to cycles.
Description:
  The DegToCycle routine converts the specified angle, in degrees, to the corresponding angle in cycles.
Parameters:
  Degrees - The angle, in degrees, to convert to cycles.
Result:
  The angle after conversion to cycles.
Donator:
  Marcel van Brakel
--------------------------------------------------------------------------------
@@DegToGrad
<GROUP UnitConversions.AngleConversion>
Summary:
  Converts an angle from degrees to gradians.
Description:
  The DegToGrad routine converts the specified angle, in degrees, to the corresponding angle in gradians.
Parameters:
  Degrees - The angle, in degrees, to convert to gradians.
Result:
  The angle after conversion to gradians.
Donator:
  Marcel van Brakel
--------------------------------------------------------------------------------
@@DegToRad
<GROUP UnitConversions.AngleConversion>
Summary:
  Converts an angle from degrees to radians.
Description:
  The DegToRad routine converts the specified angle, in degrees, to the corresponding angle in radians.
Parameters:
  Degrees - The angle, in degrees, to convert to radians.
Result:
  The angle after conversion to radians.
Donator:
  Marcel van Brakel
--------------------------------------------------------------------------------
@@GradToCycle
<GROUP UnitConversions.AngleConversion>
Summary:
  Converts an angle from gradians to cycles.
Description:
  The GradToCycle routine converts the specified angle, in gradians, to the corresponding angle in cycles.
Parameters:
  Grads - The angle, in gradians, to convert to cycles.
Result:
  The angle after conversion to cycles.
Donator:
  Marcel van Brakel
--------------------------------------------------------------------------------
@@GradToDeg
<GROUP UnitConversions.AngleConversion>
Summary:
  Converts an angle from gradians to degrees.
Description:
  The GradToDeg routine converts the specified angle, in gradians, to the corresponding angle in degrees.
Parameters:
  Grads - The angle, in gradians, to convert to degrees.
Result:
  The angle after conversion to degrees.
Donator:
  Marcel van Brakel
--------------------------------------------------------------------------------
@@GradToRad
<GROUP UnitConversions.AngleConversion>
Summary:
  Converts an angle from gradians to radians.
Description:
  The GradToRad routine converts the specified angle, in gradians, to the corresponding angle in radians.
Parameters:
  Grads - The angle, in gradians, to convert to radians.
Result:
  The angle after conversion to radians.
Donator:
  Marcel van Brakel
--------------------------------------------------------------------------------
@@RadToCycle
<GROUP UnitConversions.AngleConversion>
Summary:
  Converts an angle from radians to cycles.
Description:
  The RadToCycle routine converts the specified angle, in radians, to the corresponding angle in cycles.
Parameters:
  Radians - The angle, in radians, to convert to cycles.
Result:
  The angle after conversion to cycles.
Donator:
  Marcel van Brakel
--------------------------------------------------------------------------------
@@RadToDeg
<GROUP UnitConversions.AngleConversion>
Summary:
  Converts an angle from radians to degrees.
Description:
  The RadToDeg routine converts the specified angle, in radians, to the corresponding angle in degrees.
Parameters:
  Radians - The angle, in radians, to convert to degrees.
Result:
  The angle after conversion to degrees.
Donator:
  Marcel van Brakel
--------------------------------------------------------------------------------
@@RadToGrad
<GROUP UnitConversions.AngleConversion>
Summary:
  Converts an angle from radians to gradians.
Description:
  The RadToGrad routine converts the specified angle, in radians, to the corresponding angle in gradians.
Parameters:
  Radians - The angle, in radians, to convert to gradians.
Result:
  The angle after conversion to gradians.
Donator:
  Marcel van Brakel
--------------------------------------------------------------------------------
@@DmsToDeg
<GROUP UnitConversions.AngleConversion>
Summary:
  Converts an angle given in DMS representation to degrees.
Description:
  The DmsToDeg function converts an angle given in DMS (degrees, minutes, seconds) representation
  to degrees. This is the inverse function to DegToDms.

  The M and S arguments are required to be not less than zero and not greater than 60, otherwise
  the function will yield an domain error exception.
Parameters:
  D - Arc degrees.
  M - Arc minutes.
  S - Arc seconds.
Result:
  The angle in degrees.
See also:
  DegToDms
  DmsToRad
Donator:
  Robert Rossmair
--------------------------------------------------------------------------------
@@DmsToRad
<GROUP UnitConversions.AngleConversion>
Summary:
  Converts an angle given in DMS representation to radians.
Description:
  The DmsToRad function converts an angle given in DMS (degrees, minutes, seconds) representation
  to radians.
  
  The M and S arguments are required to be not less than zero and not greater than 60, otherwise
  the function will yield an domain error exception.
Parameters:
  D - Arc degrees.
  M - Arc minutes.
  S - Arc seconds.
Result:
  Returns the angle in radians.
Donator:
  Robert Rossmair
--------------------------------------------------------------------------------
@@DegToDms@Float@Integer@Integer@Float
<GROUP UnitConversions.AngleConversion>
Summary:
  Converts an angle given in degrees to DMS representation.
Description:
  The DegToDms routine converts an angle given in arc degrees
  to DMS (degrees, minutes, seconds) representation.

  Note that the output of this routine is not intended to be
  fed to calculations (since the sign of the minute and second
  fractions is not preserved), but for display only.
Parameters:
  D - Receives the degrees.
  Degrees - The angle (in degrees) to split into integral degrees, minutes and seconds.
  M - Receives the minutes.
  S - Receives the seconds.
See also:
  DmsToDeg
  DegToDmsStr
Donator:
  Robert Rossmair
--------------------------------------------------------------------------------
@@DegToDmsStr
<GROUP UnitConversions.AngleConversion>
Summary:
  Returns a string representation of a given angle in degrees.
Description:
  The DegToDmsStr routine returns a string representation of
  the input angle as used e.g. in cartography.
Parameters:
  Degrees - The angle in degrees.
  SecondPrecision - Specifies the number of digits after the decimal point for the arc seconds
  component; default is 3 decimal digits.
Result:
  A string representation composed as follows: <arc degrees>� <arc minutes>' <arc seconds>"
Donator:
  Robert Rossmair
--------------------------------------------------------------------------------
@@CartesianToPolar
<GROUP UnitConversions.CoordinateConversion>
Summary:
  Converts cartesian to polar coordinates.
Description:
  CartesianToPolar converts the supplied coordinates from the cartesian (or
  rectangular) system to the corresponding coordinates in the polar coordinate system.
Parameters:
  X - X element of the coordinate in the cartesian system to convert.
  Y - Y element of the coordinate in the cartesian system to convert.
  Rho - Receives the Rho part of the converted coordinate.
  Theta - Receives the Theta part of the converted coordinate.
See also:
  PolarToCartesian
Donator:
  ESB Consultancy
--------------------------------------------------------------------------------
@@PolarToCartesian
<GROUP UnitConversions.CoordinateConversion>
Summary:
  Converts polar to cartesian coordinates
Description:
  PolarToCartesion converts the supplied coordinates from the polar coordinate
  system to the corresponding coordinates in the cartesian (or rectangular) system.
Parameters:
  Rho - Rho part of the coordinate in the polar system to convert.
  Theta - Theta part of the coordinate in the polar system to convert.
  X - Receives the X element of the converted coordinate.
  Y - Receives the Y element of the converted coordinate.
See also:
  CartesianToPolar
Donator:
  ESB Consultancy
--------------------------------------------------------------------------------
@@CartesianToCylinder
<GROUP UnitConversions.CoordinateConversion>
Summary:
  Converts cartesian to cylinder coordinates.
Description:
  CartesianToCylinder converts the supplied coordinates from the cartesian coordinate
  system to the corresponding coordinates in the cylinder coordinate system.
Parameters:
  X - TODO
  Y - TODO
  Z - TODO
  R - TODO
  Phi - TODO
  Zeta - TODO
Donator:
  Marcel van Brakel
--------------------------------------------------------------------------------
@@CartesianToSpheric
<GROUP UnitConversions.CoordinateConversion>
Summary:
  Converts cartesian to spheric coordinates.
Description:
  CartesianToSpheric converts the supplied coordinates from the cartesian coordinate
  system to the corresponding coordinates in the spheric coordinate system.
Parameters:
  X - X element of the coordinate in the cartesian system to convert.
  Y - Y element of the coordinate in the cartesian system to convert.
  Z - Z element of the coordinate in the cartesian system to convert.
  Rho - TODO
  Phi - TODO
  Theta - TODO
Donator:
  Marcel van Brakel
--------------------------------------------------------------------------------
@@CylinderToCartesian
<GROUP UnitConversions.CoordinateConversion>
Summary:
  Converts cylinder to cartesian coordinates.
Description:
  CylinderToCartesian converts the supplied coordinates from the cylinder coordinate
  system to the corresponding coordinates in the cartesian coordinate system.
Parameters:
  R - TODO
  Phi - TODO
  Zeta - TODO
  X - TODO
  Y - TODO
  Z - TODO
Donator:
  Marcel van Brakel
--------------------------------------------------------------------------------
@@SphericToCartesian
<GROUP UnitConversions.CoordinateConversion>
Summary:
  Converts spheric to cartesian coordinates.
Description:
  SphericToCartesian converts the supplied coordinates from the spheric coordinate
  system to the corresponding coordinates in the cartesian coordinate system.
Parameters:
  Rho - TODO
  Theta - TODO
  Phi - TODO
  X - TODO
  Y - TODO
  Z - TODO
Donator:
  Marcel van Brakel
--------------------------------------------------------------------------------
@@CmToInch
<GROUP UnitConversions.LengthConversion>
Summary:
  Converts a length from centimeters to inches.
Description:
  The CmToInch routine converts the specified length, in centimeters, to the corresponding length in inches.
Parameters:
  Cm - The length, in centimeters, to convert to inches.
Result:
  The length after conversion to inches.
Donator:
  Marcel van Brakel
--------------------------------------------------------------------------------
@@InchToCm
<GROUP UnitConversions.LengthConversion>
Summary:
  Converts a length from inches to centimeters.
Description:
  The InchToCm routine converts the specified length, in inches, to the corresponding length in centimeters.
Parameters:
  Inch - The length, in inches, to convert to centimeters.
Result:
  The length after conversion to centimeters.
Donator:
  Marcel van Brakel
--------------------------------------------------------------------------------
@@FeetToMetre
<GROUP UnitConversions.LengthConversion>
Summary:
  Converts a length from feet to meters.
Description:
  The FeetToMetre routine converts the specified length, in feet, to the corresponding length in meters.
Parameters:
  Feet - The length, in feet, to convert to meters.
Result:
  The length after conversion to meters.
Donator:
  Marcel van Brakel
--------------------------------------------------------------------------------
@@MetreToFeet
<GROUP UnitConversions.LengthConversion>
Summary:
  Converts a length from meters to feet.
Description:
  The MetreToFeet routine converts the specified length, in meters, to the corresponding length in feet.
Parameters:
  Metre - The length, in meters, to convert to feet.
Result:
  The length after conversion to feet.
Donator:
  Marcel van Brakel
--------------------------------------------------------------------------------
@@YardToMetre
<GROUP UnitConversions.LengthConversion>
Summary:
  Converts a length from yards to meters.
Description:
  The YardToMetre routine converts the specified length, in yards, to the corresponding length in meters.
Parameters:
  Yard - The length, in yards, to convert to meters.
Result:
  The length after conversion to meters.
Donator:
  Marcel van Brakel
--------------------------------------------------------------------------------
@@MetreToYard
<GROUP UnitConversions.LengthConversion>
Summary:
  Converts a length from meters to yards.
Description:
  The MetreToYard routine converts the specified length, in meters, to the corresponding length in yards.
Parameters:
  Metre - The length, in meters, to convert to yards.
Result:
  The length after conversion to yards.
Donator:
  Marcel van Brakel
--------------------------------------------------------------------------------
@@NmToKm
<GROUP UnitConversions.LengthConversion>
Summary:
  Converts a length from nautic miles to kilometers.
Description:
  The NmToKm routine converts the specified length, in nautic miles, to the corresponding length in kilometers.
Parameters:
  Nm - The length, in nautic miles, to convert to kilometers.
Result:
  The length after conversion to kilometers.
Donator:
  Marcel van Brakel
--------------------------------------------------------------------------------
@@KmToNm
<GROUP UnitConversions.LengthConversion>
Summary:
  Converts a length from kilometers to nautic miles.
Description:
  The KmToNm routine converts the specified length, in kilometers, to the corresponding length in nautic miles.
Parameters:
  Km - The length, in kilometers, to convert to nautic miles.
Result:
  The length after conversion to nautic miles.
Donator:
  Marcel van Brakel
--------------------------------------------------------------------------------
@@KmToSm
<GROUP UnitConversions.LengthConversion>
Summary:
  Converts a length from kilometers to statute miles.
Description:
  The KmToSm routine converts the specified length, in kilometers, to the corresponding length in statute miles.
Parameters:
  Km - The length, in kilometers, to convert to statute miles.
Result:
  The length after conversion to statute miles.
Donator:
  Marcel van Brakel
--------------------------------------------------------------------------------
@@SmToKm
<GROUP UnitConversions.LengthConversion>
Summary:
  Converts a length from statute miles to kilometers.
Description:
  The SmToKm routine converts the specified length, in statute miles, to the corresponding length in kilometers.
Parameters:
  Sm - The length, in statute miles, to convert to kilometers.
Result:
  The length after conversion to kilometers.
Donator:
  Marcel van Brakel
--------------------------------------------------------------------------------
@@PascalToBar
<GROUP UnitConversions.PressureConversion>
Summary:
  Converts a pressure from Pascal to bar.
Description:
  The PascalToBar routine converts the specified pressure, in Pascal, to the corresponding pressure in bar.
Parameters:
  Pa - The pressure, in Pascal, to convert to bar.
Result:
  The pressure after conversion to bar.
Donator:
  Matthias Thoma
--------------------------------------------------------------------------------
@@PascalToAt
<GROUP UnitConversions.PressureConversion>
Summary:
  Converts a pressure from Pascal to at.
Description:
  The PascalToAt routine converts the specified pressure, in Pascal, to the corresponding pressure in at.
Parameters:
  Pa - The pressure, in Pascal, to convert to at.
Result:
  The pressure after conversion to at.
Donator:
  Matthias Thoma
--------------------------------------------------------------------------------
@@PascalToTorr
<GROUP UnitConversions.PressureConversion>
Summary:
  Converts a pressure from Pascal to torr.
Description:
  The PascalToTorr routine converts the specified pressure, in Pascal, to the corresponding pressure in torr.
Parameters:
  Pa - The pressure, in Pascal, to convert to torr.
Result:
  The pressure after conversion to torr.
Donator:
  Matthias Thoma
--------------------------------------------------------------------------------
@@BarToPascal
<GROUP UnitConversions.PressureConversion>
Summary:
  Converts a pressure from bar to Pascal.
Description:
  The BarToPascal routine converts the specified pressure, in bar, to the corresponding pressure in Pascal.
Parameters:
  Bar - The pressure, in bar, to convert to Pascal.
Result:
  The pressure after conversion to Pascal.
Donator:
  Matthias Thoma
--------------------------------------------------------------------------------
@@AtToPascal
<GROUP UnitConversions.PressureConversion>
Summary:
  Converts a pressure from at to Pascal.
Description:
  The AtToPascal routine converts the specified pressure, in at, to the corresponding pressure in Pascal.
Parameters:
  At - The pressure, in at, to convert to Pascal.
Result:
  The pressure after conversion to Pascal.
Donator:
  Matthias Thoma
--------------------------------------------------------------------------------
@@TorrToPascal
<GROUP UnitConversions.PressureConversion>
Summary:
  Converts a pressure from torr to Pascal.
Description:
  The TorrToPascal routine converts the specified pressure, in torr, to the corresponding pressure in Pascal.
Parameters:
  Torr - The pressure, in torr, to convert to Pascal.
Result:
  The pressure after conversion to Pascal.
Donator:
  Matthias Thoma
--------------------------------------------------------------------------------
@@LitreToGalUs
<GROUP UnitConversions.VolumeConversion>
Summary:
  Converts a volume from litres to US Gallons.
Description:
  The LitreToGalUs routine converts the specified volume, in litres, to the corresponding volume in US Gallons.
Parameters:
  Litre - The volume, in litres, to convert to US Gallons.
Result:
  The volume after conversion to US Gallons.
Donator:
  Marcel van Brakel
--------------------------------------------------------------------------------
@@GalUsToLitre
<GROUP UnitConversions.VolumeConversion>
Summary:
  Converts a volume from US Gallons to Litres.
Description:
  The GalUsToLitre routine converts the specified volume, in US Gallons, to the corresponding volume in Litres.
Parameters:
  GalUs - The volume, in US Gallons, to convert to Litres.
Result:
  The volume after conversion to Litres.
Donator:
  Marcel van Brakel
--------------------------------------------------------------------------------
@@GalUsToGalCan
<GROUP UnitConversions.VolumeConversion>
Summary:
  Converts a volume from US Gallons to Canadian Gallons.
Description:
  The GalUsToGalCan routine converts the specified volume, in US Gallons, to the corresponding volume in Canadian Gallons.
Parameters:
  GalUs - The volume, in US Gallons, to convert to Canadian Gallons.
Result:
  The volume after conversion to Canadian Gallons.
Donator:
  Marcel van Brakel
--------------------------------------------------------------------------------
@@GalCanToGalUs
<GROUP UnitConversions.VolumeConversion>
Summary:
  Converts a volume from Canadian Gallons to US Gallons.
Description:
  The GalCanToGalUs routine converts the specified volume, in Canadian Gallons, to the corresponding volume in US Gallons.
Parameters:
  GalCan - The volume, in Canadian Gallons, to convert to US Gallons.
Result:
  The volume after conversion to US Gallons.
Donator:
  Marcel van Brakel
--------------------------------------------------------------------------------
@@GalUsToGalUk
<GROUP UnitConversions.VolumeConversion>
Summary:
  Converts a volume from US Gallons to UK Gallons.
Description:
  The GalUsToGalUk routine converts the specified volume, in US Gallons, to the corresponding volume in UK Gallons.
Parameters:
  GalUs - The volume, in US Gallons, to convert to UK Gallons.
Result:
  The volume after conversion to UK Gallons.
Donator:
  Marcel van Brakel
--------------------------------------------------------------------------------
@@GalUkToGalUs
<GROUP UnitConversions.VolumeConversion>
Summary:
  Converts a volume from UK Gallons to US Gallons.
Description:
  The GalUkToGalUs routine converts the specified volume, in UK Gallons, to the corresponding volume in US Gallons.
Parameters:
  GalUk - The volume, in UK Gallons, to convert to US Gallons.
Result:
  The volume after conversion to US Gallons.
Donator:
  Marcel van Brakel
--------------------------------------------------------------------------------
@@LitreToGalCan
<GROUP UnitConversions.VolumeConversion>
Summary:
  Converts a volume from litres to Canadian Gallons.
Description:
  The LitreToGalCan routine converts the specified volume, in litres, to the corresponding volume in Canadian Gallons.
Parameters:
  Litre - The volume, in litres, to convert to Canadian Gallons.
Result:
  The volume after conversion to Canadian Gallons.
Donator:
  Allan Lyons
--------------------------------------------------------------------------------
@@GalCanToLitre
<GROUP UnitConversions.VolumeConversion>
Summary:
  Converts a volume from Canadian Gallons to Litres.
Description:
  The GalCanToLitre routine converts the specified volume, in Canadian Gallons, to the corresponding volume in Litres.
Parameters:
  GalCan - The volume, in Canadian Gallons, to convert to Litres.
Result:
  The volume after conversion to Litres.
Donator:
  Allan Lyons
--------------------------------------------------------------------------------
@@LitreToGalUk
<GROUP UnitConversions.VolumeConversion>
Summary:
  Converts a volume from litres to UK Gallons.
Description:
  The LitreToGalUk routine converts the specified volume, in litres, to the corresponding volume in UK Gallons.
Parameters:
  Litre - The volume, in litres, to convert to UK Gallons.
Result:
  The volume after conversion to UK Gallons.
Donator:
  Allan Lyons
--------------------------------------------------------------------------------
@@GalUkToLitre
<GROUP UnitConversions.VolumeConversion>
Summary:
  Converts a volume from UK Gallons to Litres.
Description:
  The GalUkToLitre routine converts the specified volume, in UK Gallons, to the corresponding volume in Litres.
Parameters:
  GalUs - The volume, in UK Gallons, to convert to Litres.
Result:
  The volume after conversion to Litres.
Donator:
  Allan Lyons
--------------------------------------------------------------------------------
@@KgToLb
<GROUP UnitConversions.MassConversion>
Summary:
  Converts a mass from kilograms to pounds.
Description:
  The KgToLb routine converts the specified mass, in kilograms, to the corresponding mass in pounds.
Parameters:
  Kg - The mass, in kilograms, to convert to pounds.
Result:
  The mass after conversion to pounds.
Donator:
  Marcel van Brakel
--------------------------------------------------------------------------------
@@KgToKarat
<GROUP UnitConversions.MassConversion>
Summary:
  Converts a mass from kilograms to karat.
Description:
  The KgToKarat routine converts the specified mass, in kilograms, to the corresponding
  mass in karat.
Parameters:
  Kg - The mass, in kilograms, to convert to karat.
Result:
  The mass after conversion to karat.
Donator:
  Marcel van Brakel
--------------------------------------------------------------------------------
@@LbToKg
<GROUP UnitConversions.MassConversion>
Summary:
  Converts a mass from pounds to kilograms.
Description:
  The LbToKg routine converts the specified mass, in pounds, to the corresponding mass in kilograms.
Parameters:
  Kg - The mass, in pounds, to convert to kilograms.
Result:
  The mass after conversion to kilograms.
Donator:
  Marcel van Brakel
--------------------------------------------------------------------------------
@@KgToOz
<GROUP UnitConversions.MassConversion>
Summary:
  Converts a mass from kilograms to ounce.
Description:
  The KgToOz routine converts the specified mass, in kilograms, to the corresponding mass in ounces.
Parameters:
  Kg - The mass, in kilograms, to convert to ounces.
Result:
  The mass after conversion to ounces.
Donator:
  Marcel van Brakel
--------------------------------------------------------------------------------
@@OzToKg
<GROUP UnitConversions.MassConversion>
Summary:
  Converts a mass from pounds to kilograms.
Description:
  The OzToKg routine converts the specified mass, in ounces, to the corresponding mass in kilograms.
Parameters:
  Kg - The mass, in ounces, to convert to kilograms.
Result:
  The mass after conversion to kilograms.
Donator:
  Marcel van Brakel
--------------------------------------------------------------------------------
@@QrUsToKg
<GROUP UnitConversions.MassConversion>
Summary:
  Converts a mass from quarter(US) to kilograms.
Description:
  The QrUsToKg routine converts the specified mass, in quarters, to the corresponding mass in kilograms.
Parameters:
  nQr - The mass, in quarters, to convert to kilograms.
Result:
  The mass after conversion to kilograms.
Donator:
  Marcel van Brakel
--------------------------------------------------------------------------------
@@QrUkToKg
<GROUP UnitConversions.MassConversion>
Summary:
  Converts a mass from quarter(UK) to kilograms.
Description:
  The QrUkToKg routine converts the specified mass, in quarters, to the corresponding mass in kilograms.
Parameters:
  nQr - The mass, in quarters, to convert to kilograms.
Result:
  The mass after conversion to kilograms.
Donator:
  Marcel van Brakel
--------------------------------------------------------------------------------
@@KaratToKg
<GROUP UnitConversions.MassConversion>
Summary:
  Converts a mass from karat to kilograms.
Description:
  The KaratToKg routine converts the specified mass, in karat, to the corresponding mass in kilograms.
Parameters:
  nQr - The mass, in karat, to convert to kilograms.
Result:
  The mass after conversion to kilograms.
Donator:
  Marcel van Brakel
--------------------------------------------------------------------------------
@@CwtUsToKg
<GROUP UnitConversions.MassConversion>
Summary:
  Converts a mass from centweight to kilograms.
Description:
  The CwtUsToKg routine converts the specified mass, in centweight, to the corresponding mass in kilograms.
Parameters:
  nCwt - The mass, in centweight, to convert to kilograms.
Result:
  The mass after conversion to kilograms.
Donator:
  Marcel van Brakel
--------------------------------------------------------------------------------
@@CwtUkToKg
<GROUP UnitConversions.MassConversion>
Summary:
  Converts a mass from centweight to kilograms.
Description:
  The CwtUkToKg routine converts the specified mass, in centweight, to the corresponding mass in kilograms.
Parameters:
  nCwt - The mass, in centweight, to convert to kilograms.
Result:
  The mass after conversion to kilograms.
Donator:
  Marcel van Brakel
--------------------------------------------------------------------------------
@@StonToKg
<GROUP UnitConversions.MassConversion>
Summary:
  Converts a mass from short ton to kilograms.
Description:
  The StonToKg routine converts the specified mass, in short ton, to the corresponding mass in kilograms.
Parameters:
  nSTon - The mass, in short ton, to convert to kilograms.
Result:
  The mass after conversion to kilograms.
Donator:
  Marcel van Brakel
--------------------------------------------------------------------------------
@@LtonToKg
<GROUP UnitConversions.MassConversion>
Summary:
  Converts a mass from long ton to kilograms.
Description:
  The LtonToKg routine converts the specified mass, in long ton, to the corresponding mass in kilograms.
Parameters:
  nLTon - The mass, in long ton, to convert to kilograms.
Result:
  The mass after conversion to kilograms.
Donator:
  Marcel van Brakel
--------------------------------------------------------------------------------
@@KgToCwtUs
<GROUP UnitConversions.MassConversion>
Summary:
  Converts a mass from kilograms to centweight.
Description:
  The KgToCwtUs routine converts the specified mass, in kilograms, to the corresponding mass in centweight.
Parameters:
  Kg - The mass, in kilograms, to convert to centweight.
Result:
  The mass after conversion to centweight.
Donator:
  Marcel van Brakel
--------------------------------------------------------------------------------
@@KgToCwtUk
<GROUP UnitConversions.MassConversion>
Summary:
  Converts a mass from kilograms to centweight.
Description:
  The KgToCwtUk routine converts the specified mass, in kilograms, to the corresponding mass in centweight.
Parameters:
  Kg - The mass, in kilograms, to convert to centweight.
Result:
  The mass after conversion to centweight.
Donator:
  Marcel van Brakel
--------------------------------------------------------------------------------
@@KgToQrUs
<GROUP UnitConversions.MassConversion>
Summary:
  Converts a mass from kilograms to quarter.
Description:
  The KgToQrUs routine converts the specified mass, in kilograms, to the corresponding mass in quarter.
Parameters:
  Kg - The mass, in kilograms, to convert to quarter.
Result:
  The mass after conversion to quarter.
Donator:
  Marcel van Brakel
--------------------------------------------------------------------------------
@@KgToQrUk
<GROUP UnitConversions.MassConversion>
Summary:
  Converts a mass from kilograms to quarter.
Description:
  The KgToQrUk routine converts the specified mass, in kilograms, to the corresponding mass in quarter.
Parameters:
  Kg - The mass, in kilograms, to convert to quarter.
Result:
  The mass after conversion to quarter.
Donator:
  Marcel van Brakel
--------------------------------------------------------------------------------
@@KgToSton
<GROUP UnitConversions.MassConversion>
Summary:
  Converts a mass from kilograms to short ton.
Description:
  The KgToSton routine converts the specified mass, in kilograms, to the corresponding mass in short ton.
Parameters:
  Kg - The mass, in kilograms, to convert to short ton.
Result:
  The mass after conversion to short ton.
Donator:
  Marcel van Brakel
--------------------------------------------------------------------------------
@@KgToLton
<GROUP UnitConversions.MassConversion>
Summary:
  Converts a mass from kilograms to long ton.
Description:
  The KgToLton routine converts the specified mass, in kilograms, to the corresponding mass in long ton.
Parameters:
  Kg - The mass, in kilograms, to convert to long ton.
Result:
  The mass after conversion to long ton.
Donator:
  Marcel van Brakel
--------------------------------------------------------------------------------
@@KnotToMs
<GROUP UnitConversions.Velocity>
Summary:
  Converts a velocity from knot to meters per second.
Description:
  The KnotToMs routine converts the specified velocity, in knots, to the corresponding velocity in meters per second.
Parameters:
  Knot - The velocity, in Knot, to convert to meters per second.
Result:
  The velocity after conversion to meters per second.
Donator:
  Manlio Laschena
--------------------------------------------------------------------------------
@@HpElectricToWatt
<GROUP UnitConversions.Power>
Summary:
  Converts a power from horsepower to Watt.
Description:
  The HpElectricToWatt routine converts the specified power, in horsepower, to the corresponding power in Watt.
Parameters:
  HpE - The power, in horsepower, to convert to Watt.
Result:
  The power after conversion to Watt.
Donator:
  Manlio Laschena
--------------------------------------------------------------------------------
@@HpMetricToWatt
<GROUP UnitConversions.Power>
Summary:
  Converts a power from metric horsepower to Watt.
Description:
  The HpMetricToWatt routine converts the specified power, in metric horsepower, to the corresponding power in Watt.
Parameters:
  HpM - The power, in metric horsepower, to convert to Watt.
Result:
  The power after conversion to Watt.
Donator:
  Manlio Laschena
--------------------------------------------------------------------------------
@@MsToKnot
<GROUP UnitConversions.Velocity>
Summary:
  Converts a velocity from meters per second to knots.
Description:
  The MsToKnot routine converts the specified velocity, in meters per second, to the corresponding velocity in knots.
Parameters:
  ms - The velocity, in meters per second, to convert to knots.
Result:
  The velocity after conversion to knots.
Donator:
  Manlio Laschena
--------------------------------------------------------------------------------
@@WattToHpElectric
<GROUP UnitConversions.Power>
Summary:
  Converts a power from Watt to horsepower.
Description:
  The WattToHpElectric routine converts the specified power, in Watt, to the corresponding power in horsepower.
Parameters:
  W - The power, in Watt, to convert to horsepower.
Result:
  The power after conversion to horsepower.
Donator:
  Manlio Laschena
--------------------------------------------------------------------------------
@@WattToHpMetric
<GROUP UnitConversions.Power>
Summary:
  Converts a power from Watt to metric horsepower.
Description:
  The WattToHpMetric routine converts the specified power, in Watt, to the corresponding power in metric horsepower.
Parameters:
  W - The power, in Watt, to convert to metric horsepower.
Result:
  The power after conversion to metric horsepower.
Donator:
  Manlio Laschena
--------------------------------------------------------------------------------
@@CelsiusAbsoluteZero
  This constant is representating the absolute zero point in degree Celsius.
-------------------------------------------------------------------------------
@@KelvinAbsoluteZero
  This constant is representating the absolute zero point in degree Kelvin.
-------------------------------------------------------------------------------
@@FahrenheitAbsoluteZero
  This constant is representating the absolute zero point in degree Fahrenheit.
-------------------------------------------------------------------------------
@@CelsiusBoilingPoint
  This constant is representating the boiling point of water in degree Celsius.
-------------------------------------------------------------------------------
@@KelvinBoilingPoint
  This constant is representating the boiling point of water in degree Kelvin.
-------------------------------------------------------------------------------
@@FahrenheitBoilingPoint
  This constant is representating the boiling point of water in degree Fahrenheit.
-------------------------------------------------------------------------------
@@CelsiusFreezingPoint
  This constant is representating the freezing point of water in degree Celsius.
-------------------------------------------------------------------------------
@@KelvinFreezingPoint
  This constant is representating the freezing point of water in degree Kelvin.
-------------------------------------------------------------------------------
@@FahrenheitFreezingPoint
  This constant is representating the freezing point of water in degree Fahrenheit.
-------------------------------------------------------------------------------